\hypertarget{dir_1503eef17601e40cb1ebfe4a17f336ee}{}\section{E\+:/\+React\+O\+S-\/0.4.6/dll/keyboard/kbda3 目录参考}
\label{dir_1503eef17601e40cb1ebfe4a17f336ee}\index{E\+:/\+React\+O\+S-\/0.\+4.\+6/dll/keyboard/kbda3 目录参考@{E\+:/\+React\+O\+S-\/0.\+4.\+6/dll/keyboard/kbda3 目录参考}}
